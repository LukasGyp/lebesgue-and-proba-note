\documentclass{article}

\usepackage{amsmath, amsthm, mathtools}
\usepackage{mathrsfs}
\usepackage{color}
\newcommand{\red}[1]{\textcolor{red}{#1}}

\newtheoremstyle{mystyle}
    {3pt}
    {3pt}
    {\itshape}
    {3pt}
    {\bfseries}
    {.}
    {\newline}
    {\thmname{#1}\thmnumber{ #2}\thmnote{ (#3)}}
\theoremstyle{mystyle}
\newtheorem{dfn}{Def}[section]
\newtheorem{prop}[dfn]{Prop}
\newtheorem{lem}[dfn]{Lem}
\newtheorem{thm}[dfn]{Thm}
\newtheorem{cor}[dfn]{Cor}
\newtheorem{rem}[dfn]{Rmk}

\begin{document}
\title{Lebesgue Integration and Probabilities}
\author{Luke OGURO}
\maketitle

\tableofcontents

\section{Measures on a product space}
\subsection{Product of measurable spaces}
\begin{dfn}[Product of measurable spaces]
  $(E_1, \mathcal{E}_1), \dots, (E_n, \mathcal{E}_n)$: measurable sp.\\
  Then 
  \begin{equation}
    \mathcal{E}_1 \times \cdots \times \mathcal{E}_n \coloneqq 
    \sigma(\{A_1 \times \cdots \times A_n \mid A_i \in \mathcal{E}_i, i=1, \dots, n\}) 
  \end{equation}
  is a $\sigma$-algebra on $E_1 \times \cdots \times E_n$. 
  $(E_1 \times \cdots \times E_n, \mathcal{E}_1 \times \cdots \times \mathcal{E}_n)$ 
  is called the \red{product of measurable spaces}.
\end{dfn}

\bibliographystyle{plain}
\bibliography{references}

\end{document}